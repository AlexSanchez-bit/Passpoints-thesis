\begin{conclusions}
En este trabajo se presentó una implementación propia del sistema \textit{Passpoints}, con la intención de contar con una alternativa práctica a las tradicionales contraseñas alfanuméricas. Se desarrolló una aplicación de notas privadas, utilizando \textit{Supabase} y \textit{Vue.js}, en la que el método de autenticación se basa en el citado sistema. Se mostró por qué constituye un método novedoso, superior en cuanto a seguridad y usabilidad respecto a los sistemas actuales.

Se realizó un estudio riguroso del sistema, definiendo sus características, funcionamiento y niveles de seguridad. Para ello, se analizaron y compararon los distintos métodos de discretización existentes, seleccionándose la discretización óptima por presentar la menor complejidad algorítmica y por su capacidad de generar una discretización de la imagen centrada en el punto seleccionado por el usuario, lo cual aporta mayor seguridad y reduce los errores de dicretizaci\'on.

Una de las ventajas de contar con esta implementación propia es la posibilidad de realizar experimentos con datos y usuarios reales, en contraste con las simulaciones empleadas en la mayoría de los antecedentes del uso de \textit{Passpoints}. En este sentido, se sometió el sistema a pruebas con 36 usuarios, efectuándose ataques de diccionario basados en patrones comunes (\textit{DIAG} y \textit{LINE}) sin lograr vulnerar ninguna contraseña. Además, se descartó el ataque de fuerza bruta debido a la inviabilidad que impone el extenso tamaño del espacio de contraseñas.

Por otro lado, se ha implementado un sistema de recolección de datos que almacena información relevante de los usuarios (edad, sexo y nivel educativo), junto con los intentos de autenticación, sean estos exitosos o fallidos. Esto permitirá en el futuro analizar correlaciones y tendencias en la selección de puntos e imágenes, diferenciando el comportamiento de un usuario legítimo del de uno no autorizado, y sentando las bases para la incorporación de tests que detecten contraseñas de baja aleatoriedad.

\end{conclusions}
