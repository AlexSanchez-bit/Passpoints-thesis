\begin{conclusions}
   En  este  trabajo  se  presentó  una  implementación  propia  del  sistema  \textit{Passpoints},  con  la  intención  de  contar  con  una alternativa práctica a las tradicionales contraseñas alfanuméricas. Se mostró por qué constituye un método novedoso superior en cuanto a seguridad y usabilidad con respecto a los sistemas actuales basados en contraseñas alfanuméricas. Se  realizó  un  estudio  riguroso  de  este  sistema,  por  lo  que  fueron  definidas  sus  características,  funcionamiento  y seguridad.  Se  analizaron  y  compararon  los  distintos  métodos  de  discretización  existentes,  seleccionando  la discretización optimal por presentar la menor complejidad algorítmica y por ser una de las dos discretizaciones que ofrecen  mayor  seguridad.  Una  de  las  ventajas  de  contar  con  esta  implementación  propia  será  poder  realizar experimentos  empleando  datos  y  usuarios  reales,  y  no simulaciones  como  se  había  hecho  hasta  el  momento  en  la mayoría de los antecedentes que hacían uso del \textit{Passpoints}.
\end{conclusions}
