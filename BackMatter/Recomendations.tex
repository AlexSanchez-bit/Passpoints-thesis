\begin{recomendations}
   A partir de los resultados obtenidos en este trabajo y con el objetivo de continuar el desarrollo y mejora del sistema \textit{Passpoints}, se sugieren las siguientes recomendaciones para trabajos futuros:
   
   \begin{itemize}
   	\item 	\textbf{Análisis de correlación entre datos de usuarios y aleatoriedad de contraseñas}: Se recomienda estudiar la relación entre los datos recopilados de los usuarios y la aleatoriedad de las contraseñas gráficas que estos generan. Este análisis permitirá evaluar si existen patrones predecibles que puedan comprometer la seguridad del sistema.
   	
   	\item 	\textbf{Implementación de \textit{tests} de aleatoriedad para contraseñas gráficas}: Se sugiere incorporar pruebas que permitan detectar patrones predecibles en las contraseñas gráficas. Estos \textit{tests}, contribuirán a mejorar la robustez del sistema contra ataques de ingeniería social y fuerza bruta.
   	
   	\item 	\textbf{Mecanismo de detección de intentos de autenticación}: Se recomienda desarrollar un mecanismo capaz de reconocer los intentos de autenticación de un usuario legítimo y descartar aquellos provenientes de un atacante. Un posible enfoque para esta implementación es el modelo propuesto en \cite{legon2019nuevo}, el cual podría ser adaptado y evaluado en el contexto del sistema 	extit{Passpoints}.
   	
   	\item 	\textbf{Pruebas a mayor escala}: Para validar la viabilidad del sistema en escenarios reales, se sugiere realizar pruebas con una mayor cantidad de usuarios. Esto permitirá obtener datos más representativos y evaluar el rendimiento y la seguridad del sistema en entornos con alta concurrencia.
   	
   	\item 	\textbf{Uso de imágenes de mayor dimensión}: Se recomienda emplear imágenes de mayor resolución en el sistema, lo que podría incrementar el espacio de contrase\~nas y, en consecuencia, mejorar la seguridad general del método.
   	
   	\item 	\textbf{Ampliación del conjunto de imágenes}: Se sugiere incorporar una cantidad significativa de imágenes al sistema con el fin de diversificar las opciones disponibles para los usuarios y reducir la posibilidad de selección de puntos predecibles dentro de las mismas.
   	
   \end{itemize}
\end{recomendations}
