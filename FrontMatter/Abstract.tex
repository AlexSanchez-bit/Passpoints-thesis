\begin{resumen}
	La autenticación es crucial para la protección de los usuarios y sus datos. Debido a las debilidades que aparecen en las contraseñas alfanuméricas por la acción de los usuarios, se han desarrollado nuevos enfoques como son los basados en autenticación gráfica. Uno de estos sistemas es el \textit{Passpoints}, el cual se destaca por su seguridad y facilidad de uso. En este trabajo se presenta una implementación propia de dicho sistema, resultado de un exhaustivo estudio del sistema en cuestión. Dicho estudio abordó tanto el funcionamiento como la seguridad del sistema \textit{Passpoints}, identificando sus debilidades y explorando las propuestas existentes para mitigarlas. Para la implementación principal de este sistema, se llevaron a cabo otras implementaciones intermedias esenciales para su desarrollo completo. Para ello se realizó un análisis exhaustivo de los métodos de discretización disponibles con el fin de seleccionar el más efectivo y eficiente para su posterior traducción a código de programación. Así como una investigación referente a la adaptación de este sistema a la variedad de resoluciones de pantalla y tamaños de imagen actuales, permitiendo la adaptación de esta implementación a cualquier tipo de dispositivo. Este proceso es fundamental para convertir el sistema en un producto real que pueda ser evaluado por usuarios reales en diferentes medios.
\end{resumen}

\begin{abstract}
  Authentication is crucial for protecting users and their data. Due to the weaknesses that appear in alphanumeric passwords as a result of user actions, new approaches have been developed, such as those based on graphical authentication. One of these systems is Passpoints, which stands out for its security and ease of use. This work presents our own implementation of this system, the result of an exhaustive study of the system in question. This study addressed both the functioning and security of the Passpoints system, identifying its weaknesses and exploring existing proposals to mitigate them. For the main implementation of this system, other essential intermediate implementations were carried out for its complete development. To achieve this, a comprehensive analysis of available discretization methods was conducted to select the most effective and efficient for subsequent translation into programming code, as well as research regarding the adaptation of this system to the variety of current screen resolutions and image sizes, allowing the adaptation of this implementation to any type of device. This process is fundamental to turning the system into a real product that can be evaluated by real users across different media.
\end{abstract}