\begin{opinion}
	
   Los tradicionales sistemas de autenticación alfanuméricos están llamados a desaparecer. Su inherente contradicción entre seguridad y usabilidad los ha vuelto obsoletos en un mundo en constante transformación digital. El mundo debe y está avanzando hacia sistemas más seguros y amigables con el usuario. Grandes compañías tecnológicas como Google, Apple y Microsoft ya han dejado de usar estos métodos en sus servicios, en favor de otros más avanzados como la autenticación multifactor y las biométricas. Estas alternativas son seguras, pero dependientes de \textit{hardware} adicional y no están exentas de polémicas en el uso de la información recaudada.
   
   Es en este sentido donde aparecen las contraseñas gráficas, sustentadas en la superior habilidad que tienen los seres humanos de recordar imágenes o sus elementos, en lugar de largas cadenas de caracteres aleatorios. Prometen ser una alternativa segura, fácil de usar y sin necesidad de elementos físicos adicionales. Es por ello que la tesis titulada \textit{``Implementación de una API de autenticación gráfica basada en Passpoints''}, presentada por el estudiante Alex Sánchez Saez para optar por el título de Licenciado en Ciencias de la Computación, no es solo un ejercicio investigativo de culminación de estudios. Es un sólido trabajo científico, bien estructurado, que aborda un tema relevante en el campo de la seguridad informática: la necesidad de sistemas de autenticación seguros y usables.
   
   La tesis está correctamente organizada en cuatro capítulos que guían al lector desde los conceptos teóricos fundamentales hasta la implementación práctica y la validación del sistema. El autor realiza una revisión detallada de los diferentes tipos de contraseñas gráficas propuestos en la bibliografía, concluyendo al \textit{Passpoints} como el mejor candidato en cuanto a seguridad y usabilidad. La implementación del sistema \textit{Passpoints} se hace mediante tecnologías modernas como \textit{Supabase} y \textit{Vue.js}. Esto denota un enfoque práctico y actualizado que permite la escalabilidad del sistema y su fácil integración en aplicaciones del mundo real. Para dar cumplimiento a su objetivo, el estudiante no solo requirió de habilidades de programación, sino que también abordó complejos problemas matemáticos asociados, como los métodos de discretización y la segmentación de imágenes, lo que demuestra un profundo entendimiento de la base teórica y práctica involucrada.\\
   
   La tesis incluye una validación rigurosa del sistema, tanto en términos de usabilidad como de seguridad. La realización de pruebas con usuarios reales y la construcción de ataques de diccionario basados en uno de los patrones comunes (\textit{DIAG} y \textit{LINE}), enunciados en este trabajo, son aspectos importantes que respaldan la robustez del sistema y que serán ampliados y complementados en futuras investigaciones.
   
   Contar con una implementación propia de este sistema es un valioso aporte a esta línea de investigación. Abre puertas a nuevas investigaciones y estudios de casos reales que permitan avalar la seguridad y usabilidad de este sistema con vistas a su adopción e implementación a gran escala. El resultado de esta tesis tributa al proyecto: “Pruebas estadísticas de Aleatoriedad aplicadas a la seguridad de sistemas de información”, incluido en el Programa Nacional de Ciencias Básicas y Naturales, y a las tesis de doctorado de sus tutores.
   
\end{opinion}